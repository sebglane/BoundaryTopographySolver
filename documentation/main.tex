% print or digital version
\newif\ifprint
%\printtrue
%\printfalse
\printfalse
% use microtype package or not
\newif\iffast
\fasttrue
%\fastfalse
% use tikz package or not
\newif\ifusetikz
\usetikztrue
%\usetikzfalse
%------------------------------------------------------------------------------%
%---- main packages -----------------------------------------------------------%
%------------------------------------------------------------------------------%
\documentclass[%
	paper=letter,
	twoside=false,
	fontsize=12pt,
	titlepage=false,
	listof=totoc,
	bibliography=totoc,
	listof=flat,
	cleardoublepage=empty,
	headsepline=true,
	footsepline=false,
	captions=tableheading,
	numbers=noendperiod,
%	parskip=half-,
	headings=normal,
%	draft
]{scrartcl}
% >> scrhack
%\usepackage{scrhack}
%\KOMAoptions{hyperref=false}
% >> lmodern
\usepackage{lmodern}
% >> inputenc
\usepackage[utf8]{inputenc}
% >> hyphenation of words containing hyphen
\lccode`\-=`\-
\defaulthyphenchar=127 
% >> fontenc
\usepackage[T1]{fontenc}
% >> geometry
\usepackage[letterpaper]{geometry}
\ifprint
\geometry{
	includehead=true,
	hmarginratio=1:2,
	vmarginratio=3:5,
	textwidth=170mm,
	textheight=230mm,
	headheight=20pt,
	footskip=15mm,
	bindingoffset=6mm
}
\else
\geometry{
	includehead=true,
	hmarginratio=1:1,
	vmarginratio=3:5,
	textwidth=170mm,
	textheight=230mm,
	headheight=20pt,
	footskip=15mm,
	bindingoffset=0mm
}
\fi
% >> setspace
\usepackage{setspace}
% >> microtype
\iffast
\usepackage{microtype}
\fi
% >> penalize single lines
\widowpenalty=10000
\clubpenalty=10000
\displaywidowpenalty=10000
% >> \tolerance und \emergencystretch
\tolerance 1414
\emergencystretch 1.5em
% >> \hbadness
\hbadness 1414
% >> \hfuzz \vfuzz
\hfuzz 0.3pt
\vfuzz 0.3pt
% >> \raggedbottom or \flushbottom
\raggedbottom
% >> babel
\usepackage[main=english,ngerman,greek]{babel}
% >> calc
\usepackage{calc}
% >> printlen
\usepackage{printlen}
%------------------------------------------------------------------------------%
%---- math --------------------------------------------------------------------%
%------------------------------------------------------------------------------%
% >> amsmath
\usepackage[sumlimits,intlimits]{amsmath}
\usepackage{amssymb}
\usepackage{amsthm}
\usepackage{amsxtra}
\usepackage{amsfonts}
% >> esint
\usepackage{esint}
% >> mathtools
\usepackage{mathtools}
% >> nicefrac
\usepackage{nicefrac}
% >> bm
\usepackage{bm}
% >> ragged2e
\usepackage{ragged2e}
% >> \left- und \righthyphenmin
\lefthyphenmin=3 % Standard: 2
\righthyphenmin=3 % Standard: 2
% >> siunitx
\usepackage{siunitx}
% >> stmaryrd (symbols)
\usepackage{stmaryrd}
%------------------------------------------------------------------------------%
%---- tables, list ------------------------------------------------------------%
%------------------------------------------------------------------------------%
% symbols used inside itemize
\renewcommand\labelitemi{\textbulletsmall}
\renewcommand\labelitemii{$\triangleright$}
\renewcommand\labelitemiii{$\diamond$}
\renewcommand\labelitemiv{--}
% >> array
\usepackage{array}
\setlength{\extrarowheight}{1.5pt}
% >> multirow
\usepackage{multirow}
% >> tabularx
\usepackage{tabularx}
% >> booktabs
\usepackage{booktabs}
% >> paralist
\usepackage[neverdecrease]{paralist}
% >> left margin of lists
\setdefaultleftmargin{5.0ex}{2.5ex}{2.5ex}{2.5ex}{2.5ex}{2.5ex}
% >> vertical separation of items
\setlength{\plitemsep}{2.0pt plus 1.0pt minus 0.5pt} 
% >> vertical distance of paragraphs inside an item
\setlength{\plparsep}{3.0pt plus 1.0pt minus 0.5pt}
% >> adjusted bullet symbol
\newcommand{\textbulletsmall}{%
	\raisebox{(\heightof{\textbullet}-\heightof{\small\textbullet})/2}%
	{\small\textbullet}%
}
%------------------------------------------------------------------------------%
%---- floats ------------------------------------------------------------------%
%------------------------------------------------------------------------------%
% >> top, bottom and total number of floats, default: 2,1,3
\setcounter{topnumber}{3}
\setcounter{bottomnumber}{2}
\setcounter{totalnumber}{4}
% >> text and float fractions
% minimum fraction of float for float page, default: 0.5
\renewcommand{\floatpagefraction}{0.7}
% minimum text fraction, default: 0.2
\renewcommand{\textfraction}{0.1}
% maximum fraction of floats at top of page, default: 0.7
\renewcommand{\topfraction}{0.9}
% maximum fraction of floats at bottom of page, default: 0.3
\renewcommand{\bottomfraction}{0.7}
% >> change default placement of floats
\makeatletter
\renewcommand{\fps@figure}{htbp}%
\renewcommand{\fps@table}{htbp}%
\makeatother
% >> float
\usepackage{float}
%------------------------------------------------------------------------------%
%---- graphics ----------------------------------------------------------------%
%------------------------------------------------------------------------------%
% >> xcolor
\usepackage{xcolor}
% >> graphicx
\usepackage{graphicx}
\graphicspath{{./figures/}}
% >> tikz
\usepackage{tikz}
\usetikzlibrary{
	arrows,
	shapes,
	intersections,
	positioning,
	calc,
	patterns,
%	external
	}
%\tikzexternalize[prefix=tikzexternal/]
% >> pgfplots
\usepackage{pgfplots}
\pgfplotsset{
	compat=newest,
	colormap={coldwarm}{
		rgb255(0cm)=(59,76,192);
		rgb255(0.0625cm)=(68,90,204);
		rgb255(0.125cm)=(77,104,215);
		rgb255(0.1875cm)=(87,117,225);
		rgb255(0.25cm)=(98,130,234);
		rgb255(0.3125cm)=(108,142,241);
		rgb255(0.375cm)=(119,154,247);
		rgb255(0.4375cm)=(130,165,251);
		rgb255(0.5cm)=(141,176,254);
		rgb255(0.5625cm)=(152,185,255);
		rgb255(0.625cm)=(163,194,255);
		rgb255(0.6875cm)=(174,201,253);
		rgb255(0.75cm)=(184,208,249);
		rgb255(0.8125cm)=(194,213,244);
		rgb255(0.875cm)=(204,217,238);
		rgb255(0.9375cm)=(213,219,230);
		rgb255(1cm)=(221,221,221);
		rgb255(1.0625cm)=(229,216,209);
		rgb255(1.125cm)=(236,211,197);
		rgb255(1.1875cm)=(241,204,185);
		rgb255(1.25cm)=(245,196,173);
		rgb255(1.3125cm)=(247,187,160);
		rgb255(1.375cm)=(247,177,148);
		rgb255(1.4375cm)=(247,166,135);
		rgb255(1.5cm)=(244,154,123);
		rgb255(1.5625cm)=(241,141,111);
		rgb255(1.625cm)=(236,127,99);
		rgb255(1.6875cm)=(229,112,88);
		rgb255(1.75cm)=(222,96,77);
		rgb255(1.8125cm)=(213,80,66);
		rgb255(1.875cm)=(203,62,56);
		rgb255(1.9375cm)=(192,40,47);
		rgb255(2cm)=(180,4,38);
	}
}
% >> pgfplotstable
\usepackage{pgfplotstable}
%------------------------------------------------------------------------------%
%---- hyperlinks, captions ----------------------------------------------------%
%------------------------------------------------------------------------------%
% >> subfig
\usepackage{subfig}
% >> caption
\usepackage{caption}
\captionsetup{
	font=sf,
	format=hang,
	justification=RaggedRight,
	singlelinecheck=true,
	labelfont=bf,
	textfont=rm,
	position=bottom
}
\captionsetup[subfloat]{
	font=small,
	labelfont=rm,
	textfont=rm,
	captionskip=10pt,
	nearskip=0pt,
	labelformat=parens
}
% >> color definitions
% note: use before hyperref
%		use after xcolor
\definecolor{seclinks}{rgb}{0,0,0.5}  % blau, Verweise auf Abschnitte, Floats
\definecolor{citelinks}{rgb}{0,0.5,0} % grün, Verweise auf Quellen/Literatur
\definecolor{urllinks}{rgb}{0,0,0}    % schwarz, Verweise auf externe URLs
% >> hyperref
\usepackage{hyperref}
\hypersetup{%
	%hyperfootnotes=true,
	plainpages=false,
	linktocpage=false,
	unicode=true,
	breaklinks=true
}
\ifprint
\hypersetup{%
	colorlinks=false,
	pdfborder=0 0 0,
	pdfpagelayout=TwoColumnRight
}
\else
\hypersetup{%
	colorlinks=true,             % Links werden farblich dargestellt
	linkcolor=seclinks,          % "normal internal links" > Farbdefinition s.o.
	citecolor=citelinks,         % "bibliographical cites" > Farbdefinition s.o.
	urlcolor=urllinks,           % URLs > Farbdefinition s.o.
	pdfpagelayout=OneColumn      % eine Seite, kontinuierliches Scrollen
}
\fi
% >> \...autorefname
\AtBeginDocument{\renewcommand{\figureautorefname}{Fig.}}
\AtBeginDocument{\newcommand{\subfigureautorefname}{Fig.}}
\AtBeginDocument{\renewcommand{\tableautorefname}{Tab.}}
\AtBeginDocument{\renewcommand{\equationautorefname}{Eq.}}
\AtBeginDocument{\renewcommand{\chapterautorefname}{Chapter}}
\AtBeginDocument{\renewcommand{\sectionautorefname}{Sect.}}
\AtBeginDocument{\renewcommand{\subsectionautorefname}{Sect.}}
\AtBeginDocument{\renewcommand{\subsubsectionautorefname}{Sect.}}
\AtBeginDocument{\renewcommand{\appendixautorefname}{App.}}
% >> url
\usepackage{url}
% >> bookmark
\usepackage{bookmark}
\bookmarksetup{%
	open=true,        % Bookmark-Baumstruktur ausgeklappt
	openlevel=1,      % Anzahl der Ebenen, die ausgeklappt werden
	numbered=true     % Nummerierung d. Bookmarks mit Nummern
}
% >> hypcap
\usepackage[all]{hypcap}
\renewcommand{\hypcapspace}{\baselineskip}
%------------------------------------------------------------------------------%
%---- auxiliary packages ------------------------------------------------------%
%------------------------------------------------------------------------------%
% >> pdfpages
\usepackage{pdfpages}
% >> listings
\usepackage{listings}
% >> xspace
\usepackage{xspace}
% >> \numberwithin
% note: use after loading amsmath
%       use after hyperref
\numberwithin{equation}{section}
\numberwithin{figure}{section}
\numberwithin{table}{section}
% >> blindtext
\usepackage{blindtext}
% >> biblatex
\usepackage[
backend=biber,
style=authoryear,
citestyle=numeric
]{biblatex}
\addbibresource{references.bib}
% >> ifdraft
\usepackage{ifdraft}
% >> todonotes
\usepackage{todonotes}
%------------------------------------------------------------------------------%
%---- macros ------------------------------------------------------------------%
%------------------------------------------------------------------------------%
% >> abbreviations
\newcommand{\viz}{\textit{viz.}, }
\newcommand{\ie}{\textit{i.e.}, }
\newcommand{\eg}{\textit{e.g.}, }
\newcommand{\cf}{\textit{cf.}\ }
\newcommand{\wrt}{w.r.t.\ }
\newcommand{\etc}{\textit{etc.}\ }
% >> upright latin letters
\renewcommand{\d}[1]{\ensuremath{\mathrm{d}#1} }
\newcommand{\drm}[1]{\ensuremath{\,\mathrm{d}#1}}
\newcommand{\irm}{\ensuremath{\mathrm{i}}}
% >> roman numerals
\makeatletter
\newcommand*{\rom}[1]{\expandafter\@slowromancap\romannumeral #1@}
\makeatother
% >> upright greek letters
\newcommand{\gpi}{\textrm{\greektext p}}
\newcommand{\gmu}{\textrm{\greektext m}}
\newcommand{\geta}{\textrm{\greektext h}}
\newcommand{\laplace}{\text{\textgreek{D}}}
% >> variant greek letters
\newcommand{\vr}{\varrho}
\newcommand{\vt}{\vartheta}
\newcommand{\vp}{\varphi}
% >> upright greek letters
\newcommand{\veps}{\varepsilon}
% >> basis vectors
\newcommand{\ex}{\bm{e}_x}
\newcommand{\ey}{\bm{e}_y}
\newcommand{\ez}{\bm{e}_z}
% >> references scales
\newcommand{\Bref}{B_{\mathrm{ref.}}}
\newcommand{\lref}{\ell_{\mathrm{ref.}}}
\newcommand{\vref}{v_{\mathrm{ref.}}}
\newcommand{\gref}{g_{\mathrm{ref.}}}
\newcommand{\rhoref}{\rho_{\mathrm{ref.}}}
% >> dimensionless numbers
\newcommand{\Reynolds}{\mathit{Re}}
\newcommand{\MagneticReynolds}{\mathit{Rm}}
\newcommand{\Rossby}{\mathit{Ro}}
\newcommand{\Elsasser}{\mathit{El}}
\newcommand{\Euler}{\mathit{Eu}}
\newcommand{\Froude}{\mathit{Fr}}
\newcommand{\Ekman}{\mathit{E}}
\newcommand{\Alfven}{\mathit{Al}}
\newcommand{\Stratification}{\mathit{St}}
% >> wave vector
\newcommand{\kw}{\bm{k}}
\newcommand{\kperp}{\bm{k}_{\perp}}
\newcommand{\kt}{\bm{k}_\mathrm{T}}
% >> derivatives
\newcommand{\p}{\partial}
\newcommand{\dd}[2]{\frac{\mathrm{d} #1}{\mathrm{d} #2}}
\newcommand{\ddsqr}[2]{\frac{\mathrm{d}^2 #1}{\mathrm{d} #2^2}}
\newcommand{\pd}[2]{\frac{\p #1}{\p #2}}
\newcommand{\ppd}[3]{\frac{\p^2 #1}{\p #2\p #3}}
\newcommand{\pdsqr}[2]{\frac{\p^2 #1}{\p #2^2}}
% >> big operators
\makeatletter
\DeclareRobustCommand\bigop[1]{%
	\mathop{\vphantom{\sum}\mathpalette\bigop@{#1}}\slimits@
}
\newcommand{\bigop@}[2]{%
	\vcenter{%
		\sbox\z@{$#1\sum$}%
		\hbox{\resizebox{\ifx#1\displaystyle.9\fi\dimexpr\ht\z@+\dp\z@}{!}{$\m@th#2$}}%
	}%
}
\makeatother
% >> new definition of square root, it renames \sqrt as \oldsqrt
%\let\oldsqrt\sqrt
%\def\sqrt{\mathpalette\DHLhksqrt}
%\def\DHLhksqrt#1#2{%
%	\setbox0=\hbox{$#1\oldsqrt{#2\,}$}\dimen0=\ht0
%	\advance\dimen0-0.2\ht0
%	\setbox2=\hbox{\vrule height\ht0 depth -\dimen0}%
%	{\box0\lower0.4pt\box2}}
% >> assembly operator
\newcommand{\assembly}{\bigop{\mathsf{A}}}
% >> math operators
\newcommand{\jump}[1]{\ensuremath{\left[\!\left[#1\right]\!\right]}}
\newcommand{\deter}[1]{\det{\left(\bm{#1}\right)}}
\newcommand{\trace}[1]{\operatorname{tr}\left(#1\right)}
\newcommand{\order}[1]{\ensuremath{\mathcal{O}\left(#1\right)}}
\newcommand{\abs}[1]{\ensuremath{\left| #1 \right|}}
\newcommand{\norm}[1]{\ensuremath{|\!| #1 |\!|}}
\newcommand{\scalar}[2]{\ensuremath{\left\langle #1,#2\right\rangle}}
\newcommand{\conj}[1]{\ensuremath{\overline{#1}}}
\newcommand{\cdott}{{{}\cdot\!\cdot{}\,{}}}
% >> mathematical functions
\newcommand{\expo}[1]{\ensuremath{\exp{(#1)}}}
\newcommand{\ei}[1]{\operatorname{Ei}\left(#1\right)}
%\newcommand{\si}[1]{\sin{\left(#1\right)}}
\newcommand{\co}[1]{\cos{\left(#1\right)}}
\newcommand{\ta}[1]{\tan{\left(#1\right)}}
\newcommand{\cota}[1]{\cot{\left(#1\right)}}
\newcommand{\sih}[1]{\sinh{\left(#1\right)}}
\newcommand{\coh}[1]{\cosh{\left(#1\right)}}
\newcommand{\tah}[1]{\tanh{\left(#1\right)}}
\newcommand{\siq}[1]{\sin^2{\left(#1\right)}}
\newcommand{\coq}[1]{\cos^2{\left(#1\right)}}
\newcommand{\taq}[1]{\tan^2{\left(#1\right)}}
\newcommand{\asi}[1]{\arcsin{\left(#1\right)}}
\newcommand{\aco}[1]{\arccos{\left(#1\right)}}
\newcommand{\ata}[1]{\arctan{\left(#1\right)}}
% >> spherical values
\newcommand{\sinTheta}{\sin(\theta)}
\newcommand{\cosTheta}{\cos(\theta)}
\newcommand{\sinPhi}{\sin(\varphi)}
\newcommand{\cosPhi}{\cos(\varphi)}
% >> weak form operators
\newcommand*{\inner}[2]{\left(#1, #2\right)_{\Omega}}
\newcommand*{\innerSurf}[2]{\left\langle#1, #2\right\rangle_{\Gamma}}
\newcommand*{\elliptic}[2]{\mathcal{A}(#1, #2)}
\newcommand*{\saddle}[2]{\mathcal{B}(#1, #2)}
\newcommand*{\convec}[3]{\mathcal{C}(#1, #2, #3)}
%------------------------------------------------------------------------------%
\begin{document}
% >> title
\subject{Core mantle interactions of the Earth}
\title{Mechanisms of angular moment transfer}
\author{Sebastian Glane \and Bruce Buffett}
\date{\today}
\maketitle
% >> table of contents
\tableofcontents
% >> main text
\section{Problem description}
\begin{figure}
\centering
\begin{tikzpicture}[>=stealth]
	\pgfmathsetmacro\a{0.5}
	\pgfmathsetmacro\e{1.0}
	\pgfmathsetmacro\d{0.1}
	\pgfmathsetmacro\r{0.125}
	\pgfmathsetmacro\h{2.}
	\pgfmathsetmacro\b{7.5}
	\begin{scope}[shift={(-0.8*\b,\h)}]
		\draw[thick,<->] (0,\e) node[left]{$\bm{e}_z$} |- (\e,0) node[below]{$\bm{e}_x$};
		\draw[thick] (0,0) circle[radius=\r] node[below left]{$\bm{e}_y$};
		\draw[thick] (225:\r) -- (45:\r);
		\draw[thick] (135:\r) -- (-45:\r);
	\end{scope}
	% geometry
	\draw[shift={(-0.5*\b,\h)},thick,xscale=0.125*\b,fill=gray!10] (0,0) 
		sin (1,\a) cos (2,0) sin (3,-\a) cos (4,0) 
		sin (5,\a) cos (6,0) sin (7,-\a) cos (8,0) 
		-| (8,-\h) |-(0,-\h) -| cycle;
	\draw[dashed] (-0.7*\b,\h) -- (0.5*\b,\h);
	\draw[fill] (.375*\b,\h-\a) circle[radius=2pt];
	% dimensions
	\draw[<->] (0.3*\b,0) --node[midway,left]{$h_0$} (0.3*\b,\h);
	\draw (-0.375*\b,\h+\a) -- (-0.55*\b-\d,\h+\a);
	\draw[<->] (-0.55*\b,\h) --node[midway,left]{$\hat{h}$} (-0.55*\b,\h+\a);
	\draw[<->] (0,\h+1.5*\a) --node[midway,above]{$\lambda_x$} (0.5*\b,\h+1.5*\a);
	\draw (0.5*\b,\h) -- (0.5*\b,\h+1.5*\a+\d);
	\draw (0,\h) -- (0,\h+1.5*\a+\d);
	\draw (0.6*\b,0) -- (0.6*\b,\h);
	\draw[thick] (0.6*\b+1.5*\e,0) -- (0.6*\b+\e,\h);
	\foreach \i in {0,...,5}
	{
		\draw[->] (0.6*\b,\i*0.2*\h) -- (0.6*\b+1.5*\e-\i*0.1*\e,\i*0.2*\h);
	}
	% annotations
	\draw (-\b,2.1*\h) -- (\b,2.1*\h);
	\draw (-\b,-\h) -- (\b,-\h);
	\node[draw,below,thick,rounded corners] at (0,-0.5*\h) {outer core};
	\node[draw,below,thick,rounded corners] at (0,\h+4*\a) {mantle};
	\node at (0,0.5*\h) {stratified layer};
	\draw (.375*\b,\h-\a) to[in=135,out=-60] (0.4*\b,-0.2*\h) node[draw,rounded corners,text width=10em,anchor=north west,text centered]{topography of core-mantle boundary};
	\draw[fill] (0.6*\b+\e+0.25*\e,0.5*\h) circle[radius=2pt];
	\draw (0.6*\b+\e+0.25*\e,0.5*\h) to[out=0,in=225] (0.7*\b+1.5*\e,\h) node[right]{$\bar{\rho}(\bm{x})$};
	\draw[->,thick] (-0.3*\b,\d) --node[midway,left]{$\bm{\varOmega}$} (-0.3*\b,\e+\d);
	\draw[->,thick] (-0.4*\b,\d) --node[midway,left]{$\bm{B}$} (-0.4*\b,\e+\d);
	\draw[<-,thick] (-0.2*\b,\d) --node[midway,left]{$\bm{g}$} (-0.2*\b,\e+\d);
	\draw[->,thick] (-0.5*\e,-\a) --node[midway,below]{$\bm{v}$} (0.5*\e,-\a);
\end{tikzpicture}
\caption{Sketch of the stratified fluid layer at the core-mantle boundary. }
\end{figure}

The basic equations describing the problem are given by the \textsc{Navier}-\textsc{Stokes} equations using the \textsc{Boussinesq} approximation accompanied by the magnetic induction equation. The system is considered in rotating reference frame and the constant angular velocity is given by $\bm{\varOmega}=\varOmega\ez$ whose magnitude is the angular velocity of the Earth. Furthermore, the gravitational acceleration is given by $\bm{g}=-g\ez$. The corresponding dimensionless system reads:
\begin{gather}
	\nabla\cdot(\rho\bm{v})=0\,, \qquad
	\Big(\bm{v}\cdot(\nabla\otimes\bm{v})+\frac{2}{\Rossby}\bm{\varOmega}\times\bm{v}\Big)=-\nabla P+\frac{1}{\Froude^2}\rho\bm{g}+\frac{1}{\Alfven^2}\bm{B}\cdot\nabla\otimes\bm{B}\,,\\
	\nabla\cdot\bm{B}=0\,, \qquad
	\MagneticReynolds\nabla\times\left(\bm{v}\times\bm{B}\right)+\nabla^2\bm{B}=\bm{0}\,,
\end{gather}
$\Rossby$ denotes the \textsc{Rossby}, $\Froude$ the \textsc{Froude}, $\Alfven$ the \textsc{Alfv\'en} number and $\MagneticReynolds$ the magnetic \textsc{Reynolds} number. The relations of these dimensionless number to the reference quantities are given by
\begin{equation}
	\Rossby=\frac{\vref}{\varOmega\lref}\,,\quad
	\Froude=\frac{\vref}{\sqrt{\gref\lref}}\,,\quad
	\Alfven=\frac{\vref}{v_\mathrm{A}}\,,\quad 
	\MagneticReynolds=\frac{\vref\lref}{\eta}\,,
\end{equation}
where $v_\mathrm{A}=\Bref/\sqrt{\rhoref\mu}$ is the \textsc{Alfv\'en} velocity. In this problem, the reference scales are chosen as $\lref=\lambda_x$, $\rhoref=\rho_0$, $\vref=\norm{\bar{\bm{V}}}$, $\Bref=\norm{\bar{\bm{B}}}$, $\gref=\norm{\bm{g}}$. The pressure is normalized by the dynamic pressure, \ie $p_\mathrm{ref.}=\rho_0\vref^2$.

We introduce the following short-hand notations for volume and surface integrals
\begin{equation*}
	\inner{\bm A}{\bm B} = \int\limits_{\Omega} \bm A \star \bm B\, \d V \; , \quad
	\innerSurf{\bm A}{\bm B}  = \int\limits_{\Gamma} \bm A \star \bm B\, \d A \ ,
\end{equation*}
where $\bm A \star \bm B$ represents the contraction of two tensors $\bm A$ and $\bm B$ of arbitrary rank to a scalar. Furthermore, we introduce operators related to viscosity/diffusion~($\mathcal{A}$), incompressible/pressure~($\mathcal{B}$) and convection~($\mathcal{C}$).
\begin{align*}
\begin{aligned}
	\elliptic{\bm{\phi}}{\bm{\psi}}&=\inner{\nabla\bm{\phi}}{\nabla\bm{\psi}}\,, &
	\saddle{\bm{\phi}}{\psi}&=\inner{\nabla\cdot\bm{\phi}}{\psi}\,, &
	\convec{\bm{\phi}}{\bm{\psi}}{\bm{\chi}}&=\inner{\bm{\phi} \cdot (\nabla\bm{\psi})}{\bm{\chi}}\,.
\end{aligned}
\end{align*}

\section{Weak form of the hydrodynamic part}

\begin{equation}
	F(\bm{v},P,\rho)=
	\begin{bmatrix}
		-\nabla\cdot\bm{v}\\
		\bm{v}\cdot\nabla\otimes\bm{v}+\frac{2}{\Rossby}\bm{\varOmega}\times\bm{v}+\nabla P-\nu\nabla^2\bm{v}
	\end{bmatrix}
\end{equation}

\begin{equation}
	DF(\bm{v},P;\bm{\d{v}},\d{P})
	=
	\begin{bmatrix}
		-\nabla\cdot\bm{\d{v}}\\
		\bm{\d{v}}\cdot\nabla\otimes\bm{v}+\bm{v}\cdot\nabla\otimes\bm{\d{v}}+\frac{2}{\Rossby}\bm{\varOmega}\times\bm{\d{v}}+\nabla\d{P}-\nu\nabla^2\bm{v}
	\end{bmatrix}
\end{equation}

The two steps of Newton's method read
\begin{enumerate}
	\item Determine $\bm{\d{v}}^k$ and $d{P}^k$ such that
	\begin{equation}
		DF(\bm{v}^k,P^k;\bm{\d{v}}^k,\d{P}^k)=-F(\bm{v}^k,P^k)\,.
	\end{equation}
	\item Compute the updated solution according to
	\begin{equation}
		\bm{v}^{k+1}=\bm{v}^k+\bm{\d{v}}^k\,,\quad
		P^{k+1}=P^k+\d{P}^k\,.
	\end{equation}
\end{enumerate}

Weak form of the first step


If the compressibility and momentum equations are multiplied by test functions~$q$ and $\bm{w}$. After integrating the pressure by parts, the weak form reads:
\begin{gather}
-\saddle{\bm{\d{v}}^k}{q}=\saddle{\bm{v}^k}{q}\,,\label{eqn:WeakIncompressibility} \\
\begin{multlined}
	-\saddle{\bm{w}}{\d{P}^k}+\convec{\bm{\d{v}}^k}{\bm{v}}{\bm{w}}+\convec{\bm{v}}{\bm{\d{v}}^k}{\bm{w}}+\frac{2}{\Rossby}\inner{\bm{\varOmega}\times\bm{\d{v}}^k}{\bm{w}}+\nu\elliptic{\bm{\d{v}}^k}{\bm{w}}\\
	=\saddle{\bm{w}}{P^k}-\convec{\bm{v}^k}{\bm{v}^k}{\bm{w}}-\frac{2}{\Rossby}\inner{\bm{\varOmega}\times\bm{v}^k}{\bm{w}}-\nu\elliptic{\bm{v}^k}{\bm{w}}\,,
\end{multlined}\label{eqn:WeakMomentum}
\end{gather}

\section{Weak form of the continuity equation}

\begin{equation}
	S\nabla\bar{\rho}\cdot\bm{v}+\nabla\rho'\cdot\bm{v}-\nu\nabla^2\rho=0
\end{equation}
$S=\Stratification^2\Froude^2/\Rossby^2$
\begin{equation}
	F_\rho(\bm{v},\rho')=S\nabla\bar{\rho}\cdot\bm{v}+\nabla\rho'\cdot\bm{v}-\nu\nabla^2\rho'
\end{equation}

\begin{equation}
	DF_\rho(\bm{v},\rho;\bm{\d{v}},\d{\rho'})=S\nabla\bar{\rho}\cdot\bm{\d{v}}+\nabla\rho'\cdot\bm{\d{v}}+\nabla\d{\rho'}\cdot\bm{v}-\nu\nabla^2\d{\rho'}
\end{equation}

Weak form

\begin{multline}
	S\inner{\nabla\bar{\rho}\cdot\bm{\d{v}}^k}{r}+\inner{\nabla{\rho'}^k\cdot\bm{\d{v}}^k}{r}+\inner{\nabla{\d{\rho'}}^k\cdot\bm{v}^k}{r}+\nu\elliptic{{\d{\rho'}}^k}{r}=\\
	-S\inner{\nabla\bar{\rho}\cdot\bm{v}^k}{r}-\inner{\nabla{\rho'}^k\cdot\bm{v}^k}{r}-\nu\elliptic{{\rho'}^k}{r}
\end{multline}




\cleardoublepage
% >> \bibliography
\printbibliography
%------------------------------------------------------------------------------%
\end{document}